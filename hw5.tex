\documentclass{article}
\usepackage{CJK}
\usepackage{graphicx}
\usepackage[onlyps]{altfont}
\usepackage[top=1in,bottom=1in,left=1.25in,right=1.25in]{geometry}
\usepackage[colorlinks,linkcolor=blue,anchorcolor=blue,citecolor=green]{hyperref}
\newcommand{\ud}{\mathrm{d}}
\begin{CJK}{UTF8}{gbsn}
	\author{杨梓鑫\ \ 10级物理弘毅班}
	\title{第五次计算物理作业}
	\date{学号:2010301020023}
	\begin{document}
	\maketitle
\section{Problem}
 \quad\sf *3.29 Explore the intermittency route to chaos for $r \leq 163$ in more detail. Begin by calculating $z$ as a function of time for different values of $r$. Try $r=163$ (which should be in the nonchaotic regime), and several larger  values up to $r=165$ or so. For the larger values of $r$ you should observe chaotic "hiccups" like those found in Figure 3.18. Next calculate the average time between these hiccups and study how it diverges as the transition to chaos is approached. While the idea here is easy to explain, writing a program to detect hiccups is a bit tricky. One way to accomplish this is to construct a histogram of times  between adjacent maxima in $z(t)$. In the oscillatory (nonchaotic) regime these times will all be the same. An odd value signal is a hiccup.\\

\section{Model}
\quad So the Lorenz model is mainly three equations:
\begin{eqnarray*}
\frac{\ud x}{\ud t} &=& \sigma (y-x),\\
\frac{\ud y}{\ud t} &=& -xz+rx-y,\\
\frac{\ud z}{\ud t} &=& xy-bz.
\end{eqnarray*}
\quad Following the textbook, the fixed parameters are:
\begin{eqnarray*}
\sigma &=& 10\\
b &=& \frac{8}{3}
\end{eqnarray*}
The initial conditions are $x=1, y=z=0$.\\\\

\section{Different Numerical Calculation Method}
\quad In the textbook, we are given 2 different method to do the numerical calculation, $Euler$ method and $Runge$-$Kutta$ method. To observe the accuracy of the methods on the Lorenz Model, we are going to compare the results of using both methods. \\\\
\subsection{The detail of Euler Method works as:}
1. Set the initial value of $x_0,y_0,z_0,v_{x0},v_{y0},v_{z0}$;\\
2. In every cycle, calculate:
\begin{eqnarray*}
x_{i+1} = x_{i}+v_{x_{i}}\Delta t,\qquad
y_{i+1} &=& y_{i}+v_{y_{i}}\Delta t,\qquad
z_{i+1} = z_{i}+v_{z_{i}}\Delta t;\\
v_{x_{i+1}} = \sigma (y_{i}-x_{i}),\qquad
v_{y_{i+1}} &=& -x_{i}z_{i}+rx_{i}-y_{i},\qquad
v_{z_{i+1}} = x_{i}y_{i}-bz_{i};\qquad
t_{i+1} = t_{i}+\Delta t.\\
\end{eqnarray*}

\subsection{\noindent The detail of Second-Order $Runge$-$Kutta$ Method represents:}
1. Set the initial value of $x_0,y_0,z_0,v_{x0},v_{y0},v_{z0}$;\\
2. In every cycle, calculate:
\begin{eqnarray*}
x' = x_{i} + \frac{1}{2}v_{x_{i}}\Delta t, \qquad y' &=& y_{i} + \frac{1}{2}v_{y_{i}}\Delta t, \qquad z' = z_{i} + \frac{1}{2}v_{z_{i}}\Delta t;\\
v'_x = \sigma (y'-x'), \qquad v'_y &=& -x'z'+rx'-y', \qquad v'_z = x'y'-bz'.\\
x_{i+1} = x_{i}+v'_{x}\Delta t,\qquad
y_{i+1} &=& y_{i}+v'_{y}\Delta t,\qquad
z_{i+1} = z_{i}+v'_{z}\Delta t;\qquad
t_{i+1} = t_{i}+\Delta t.\\
\end{eqnarray*}

\subsection{\noindent The detail of Fourth-Order $Runge$-$Kutta$ Method represents:}
1. Set the initial value of $x_0,y_0,z_0,v_{x0},v_{y0},v_{z0}$;\\
2. In every cycle, calculate:
\begin{eqnarray*}
x^{\prime} = x_{i} + \frac{1}{2}v_{x_{i}}\Delta t, \qquad y' &=& y_{i} + \frac{1}{2}v_{y_{i}}\Delta t, \qquad z' = z_{i} + \frac{1}{2}v_{z_{i}}\Delta t;\\
v'_x = \sigma (y'-x'), \qquad v'_y &=& -x'z'+rx'-y', \qquad v'_z = x'y'-bz';\\
x^{\prime\prime} = x_{i} + \frac{1}{2}v^{\prime}_{x}\Delta t, \qquad y^{\prime\prime} &=& y_{i} + \frac{1}{2}v^{\prime}_{y}\Delta t, \qquad z^{\prime\prime} = z_{i} + \frac{1}{2}v^{\prime}_{z}\Delta t;\\
v^{\prime\prime}_x = \sigma (y^{\prime\prime}-x^{\prime\prime}), \qquad v^{\prime\prime}_y &=& -x^{\prime\prime}z^{\prime\prime}+rx^{\prime\prime}-y^{\prime\prime}, \qquad v^{\prime\prime}_z = x^{\prime\prime}y^{\prime\prime}-bz^{\prime\prime};\\
x^{\prime\prime\prime} = x_{i} + v^{\prime\prime}_{x}\Delta t, \qquad y^{\prime\prime\prime} &=& y_{i} + v^{\prime\prime}_{y}\Delta t, \qquad z^{\prime\prime\prime} = z_{i} + v^{\prime\prime}_{z}\Delta t;\\
v^{\prime\prime\prime}_x = \sigma (y^{\prime\prime\prime}-x^{\prime\prime\prime}), \qquad v^{\prime\prime\prime}_y &=& -x^{\prime\prime\prime}z^{\prime\prime\prime}+rx^{\prime\prime\prime}-y^{\prime\prime\prime}, \qquad v^{\prime\prime\prime}_z = x^{\prime\prime\prime}y^{\prime\prime\prime}-bz^{\prime\prime\prime};\\
x_{i+1} &=& x_{i}+ \frac{1}{6}\left[v_{x_i}+2v^{\prime}_x+2v^{\prime\prime}_x+v^{\prime\prime\prime}_x\right]\Delta t;\\
y_{i+1} &=& y_{i}+ \frac{1}{6}\left[v_{y_i}+2v^{\prime}_y+2v^{\prime\prime}_y+v^{\prime\prime\prime}_y\right]\Delta t;\\
z_{i+1} &=& z_{i}+ \frac{1}{6}\left[v_{z_i}+2v^{\prime}_z+2v^{\prime\prime}_z+v^{\prime\prime\prime}_z\right]\Delta t.\\
t_{i+1} = t_{i}+\Delta t.\\
\end{eqnarray*}
\subsection{Comparison}
\quad The result is in Figure 1 to 3\footnote{generated by Gnuplot}, from upper to lower rows they are $Euler$ method, second-order $Runge$-$Kutta$ method and fourth-order $Runge$-$Kutta$ method respectively. From left to right, the value of $r$ varies from $160$ to $163$.\\\\
\quad We can see that although the general shapes are alike among these six picture, with a close observation one can find out a lot difference with each method. The theorically least accurate $Euler$ method varies the most. And indeed that when $r=160$, at last only Fourth-order $Runge$-$Kutta$ method gives us a regular oscillation at the end. But when $r=163$, it is not as regular as we expect, which means the threshold value of $r$ is not $163.8$. Neither does Second-order $Runge$-$Kutta$ method, might be the different computer accuracy influence. So next step, we are going to find the threshold value of $r$ for the Fourth-order $Runge$-$Kutta$ method.
\begin{figure}[htbp]
\centering
\includegraphics[scale=0.596]{/home/alexandra/CP_Hw/Chap3_2/Euler_Method160.pdf}
\includegraphics[scale=0.596]{/home/alexandra/CP_Hw/Chap3_2/Euler_Method160.pdf}
\caption{$Euler$ method}
\end{figure}


\begin{figure}[htbp]
\centering
\includegraphics[scale=0.596]{/home/alexandra/CP_Hw/Chap3_2/2-order_RK_Method160.pdf}
\includegraphics[scale=0.596]{/home/alexandra/CP_Hw/Chap3_2/2-order_RK_Method163.pdf}
\caption{Second-order $Runge$-$Kutta$ method}
\end{figure}

\begin{figure}[htbp]
\centering
\includegraphics[scale=0.596]{/home/alexandra/CP_Hw/Chap3_2/4-order_RK_Method160.pdf}
\includegraphics[scale=0.596]{/home/alexandra/CP_Hw/Chap3_2/4-order_RK_Method163.pdf}
\caption{Fourth-order $Runge$-$Kutta$ method}
\end{figure}
\newpage
\section{The threshold value of $r$}
\quad Now focus on Fourth-order $Runge$-$Kutta$ method, we have already know that $r=160$ is in the regular regime while $r=163$ is not. So the basic strategy here is dichotomy with appropriate adjustment. In order to observe the regular oscillation the plotting time is extended to $50$ seconds. And we find out that the threshold value of $r$ is $160.8$:\\
\begin{figure}[htbp]
\centering
\includegraphics{/home/alexandra/CP_Hw/Chap3_2/4-order_RK_Method160_7.pdf}
\includegraphics{/home/alexandra/CP_Hw/Chap3_2/4-order_RK_Method160_8.pdf}
\caption{Fourth-order $Runge$-$Kutta$ method when $r=160.7$ and $160.8$}
\end{figure}
\newpage
\section{Fourier Transformation of the $z-t$ diagram}
\quad For an enhanced analysis of the hiccups and the transition to chaos, we call on $Fourier$ $Transfor$$mation$ to turn to the frequency regime, in which the irregular oscillation of hiccups is more obvious to detect. With the help of GSL\footnote{I didn't use the function TVirtualFFT in ROOT because it took too much time to run and crushed all the time for no good reason, so I got mad and abandoned it, even not feeling like using it to plot this time. So BTW, the rest of the figures, are generated by Gnuplot, too.}, firstly I did a FFT on the critical values $r=160.7$ and $r=160.8$. I have to emphasize that I have zoomed up the tail part of the diagrams so that the variation is more easily to be seen, or the global view of the whole transformation is a total crumble nasty overcrowed mess! \\
\quad An elegant and relatively simple diagram of $r=160.7$ declares its nonrandomness, while the crazy high-like line of $r=160.8$ is consitent with the random behavior in the time regime.\\
\begin{figure}[htbp]
\centering
\includegraphics{/home/alexandra/CP_Hw/Chap3_2/FFT==>>4-order_RK_Method160_7.pdf}
\includegraphics{/home/alexandra/CP_Hw/Chap3_2/FFT==>>4-order_RK_Method160_8.pdf}
\caption{Fast Fourier Transformation of $z(t)$ when $r=160.7$ and $160.8$}
\end{figure}
\newpage

Next Step, the problem askes about how the transition is approached. So in Figure 6 the in-between state is quite clear that $r=160.75$ indeed is less random than $160.8$ and more that than $160.7$. At the end of the assignment, with a larger stride, the property of $r=160$ to $163$ is plotted. \\ 
\begin{figure}[h]
\centering
\includegraphics[scale=0.35]{/home/alexandra/CP_Hw/Chap3_2/FFT==>>4-order_RK_Method160_7.pdf}
\includegraphics[scale=0.35]{/home/alexandra/CP_Hw/Chap3_2/FFT==>>4-order_RK_Method160_8.pdf}
\includegraphics{/home/alexandra/CP_Hw/Chap3_2/FFT==>>4-order_RK_Method160_75.pdf}
\caption{The transition in the exact midpoint of $r=160.7$ and $160.8$}
\end{figure}


\begin{figure}[htbp]
\centering
\end{figure}
\begin{figure}[htbp]
\centering
\includegraphics[scale=0.596]{/home/alexandra/CP_Hw/Chap3_2/FFT==>>4-order_RK_Method160.pdf}
\includegraphics[scale=0.596]{/home/alexandra/CP_Hw/Chap3_2/FFT==>>4-order_RK_Method160_2.pdf}
\includegraphics[scale=0.596]{/home/alexandra/CP_Hw/Chap3_2/FFT==>>4-order_RK_Method160_4.pdf}
\includegraphics[scale=0.596]{/home/alexandra/CP_Hw/Chap3_2/FFT==>>4-order_RK_Method160_6.pdf}
\caption{nonrandom regime}
\end{figure}
\begin{figure}[htbp]
\centering
\includegraphics[scale=0.596]{/home/alexandra/CP_Hw/Chap3_2/FFT==>>4-order_RK_Method160_8.pdf}
\includegraphics[scale=0.596]{/home/alexandra/CP_Hw/Chap3_2/FFT==>>4-order_RK_Method161.pdf}
\includegraphics[scale=0.596]{/home/alexandra/CP_Hw/Chap3_2/FFT==>>4-order_RK_Method162.pdf}
\includegraphics[scale=0.596]{/home/alexandra/CP_Hw/Chap3_2/FFT==>>4-order_RK_Method163.pdf}
\caption{random regime}
\end{figure}

\end{CJK}
\end{document}

